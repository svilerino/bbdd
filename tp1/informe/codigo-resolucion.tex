\subsection{Codigo de resolucion de consignas}
\subsubsection{Script de creacion de base de datos fisica}
Por motivos de latex y caracteres, no se pueden incluir los scripts. Los scripts .sql se encuentran en la carpeta fuentes.

\subsubsection{Consultas pedidas por el enunciado}
\begin{enumerate}
	\item Poder obtener los ganadores de las elecciones transcurridas en el último año.
		Para resolver este problema nos creamos una vista auxiliar que nos devuelve la lista de elcciones del ultimo a\~no, 
		con sus resultados, es decir, el ranking (Candidato, CantVotos) para cada eleccion del último año.

		\begin{lstlisting}
		CREATE VIEW [dbo].[Ranking_Elecciones_Cargo_Ultimo_Anio] AS
		--Vista auxiliar para ranking de las ultimas elecciones del anio.
		SELECT 
	elec.Fecha as FechaEleccion, 
	elec.tipo as TipoEleccion,
	(ciu.Nombre + ciu.Apellido) as Candidato, 
	COUNT(vc.idVoto) AS CantVotos
FROM         dbo.Voto AS v 
INNER JOIN dbo.Voto_A_Candidato AS vc ON v.idVoto = vc.idVoto -- Quiero saber a que candidato fue ese voto
INNER JOIN dbo.Ciudadano AS ciu ON vc.DNI= ciu.DNI -- Join para saber el nombre del candidato(es un ciudadano)
INNER JOIN dbo.Mesa AS m ON m.idMesa = v.idMesa -- Join para saber la eleccion de la mesa y agrupar
INNER JOIN dbo.Eleccion AS elec ON elec.idEleccion = m.idEleccion -- Join para saber la fecha de la eleccion
WHERE	(v.idMesa IN
	(SELECT     idMesa
	FROM          dbo.Mesa AS m
	WHERE      (idEleccion IN
		(SELECT     idEleccion
		FROM          dbo.Eleccion AS e
		WHERE      (YEAR(Fecha) = YEAR(GETDATE()))))))
GROUP BY vc.DNI, ciu.Nombre, ciu.Apellido, elec.Fecha, elec.tipo
ORDER BY FechaEleccion ASC, CantVotos DESC

		\end{lstlisting}
		
	Luego aplicamos una consulta sobre esta vista, que nos devuelve para cada eleccion de la vista, el (o los en caso de empate) ganador(es).
	Creamos otra vista para acceder directamente a esta consulta pedida.
		
		\begin{lstlisting}
			CREATE VIEW [dbo].[Ganadores_Elecciones_Cargo_Ultimo_Anio] AS
			SELECT FechaEleccion, Candidato, MAX(CantVotos) as ganador_con_max_votos
			FROM Ranking_Elecciones_Cargo_Ultimo_Anio
			GROUP BY FechaEleccion, Candidato
		\end{lstlisting}

		Finalmente, se accede a los datos pedidos ejecutando:
		\begin{lstlisting}
			SELECT * FROM dbo.[Ganadores_Elecciones_Cargo_Ultimo_Anio]
		\end{lstlisting}

	\item Poder consultar las cinco personas que más tarde fueron a votar antes de terminar
	la votación por cada centro electoral en una elección.
	Usamos una tabla intermedia que obtiene los votos de la eleccion, particionado por centro electoral, a su vez, utilizamos la funcion \texttt{ROW\_NUMBER} para numerar los resultados de cada grupo, ordenados por tiempo de votacion y quedarnos con los 5 mayores tiempos de votacion de cada centro.
	\begin{lstlisting}
		WITH TOPFIVE AS (
	    SELECT cen.Nombre_Establecimiento, (ciu.Nombre + ciu.Apellido) as Votante, p.selloVoto as HoraVoto,
	    ROW_NUMBER() over (
	        PARTITION BY cen.idCentro
	        ORDER BY p.selloVoto DESC
	    ) AS NumFila
	    FROM Padron p
	    INNER JOIN Mesa m ON m.idMesa = p.idMesa
	    INNER JOIN Centro cen ON cen.idCentro = m.idCentro
	    INNER JOIN Ciudadano ciu ON ciu.DNI = p.DNI
	    WHERE p.idMesa IN (SELECT m.idMesa FROM Mesa m WHERE m.idEleccion = 2)--<eleccionTarget>
		)
	SELECT Nombre_Establecimiento, Votante, HoraVoto FROM TOPFIVE WHERE NumFila <= 5--cantidad de items por grupo
	\end{lstlisting}

	\item Poder consultar quienes fueron los partidos políticos que obtuvieron más del 20\%
	en las últimas cinco elecciones provinciales a gobernador.
	\textbf{Nota:} En las elecciones a cargo provincial se elije gobernador(legislativas y municipales estan separadas de las provinciales).\\

Creamos una vista que nos devuelve el ranking de las ultimas 5 elecciones a gobernador, y ademas la cantidad de votos por cada eleccion.

\begin{lstlisting}
	SELECT elec.Fecha AS FechaEleccion, ciu.Nombre + ciu.Apellido AS Candidato, partPolitico.Nombre as PartidoPolitico, 
COUNT(vc.idVoto) AS CantVotos,  

COUNT(*) over (
	        PARTITION BY elec.fecha
	    ) AS Cant_Total_Votos_Por_Eleccion  

FROM         dbo.Voto AS v 
                      INNER JOIN dbo.Voto_A_Candidato AS vc ON v.idVoto = vc.idVoto 
                      INNER JOIN dbo.Ciudadano AS ciu ON vc.DNI = ciu.DNI 
                      INNER JOIN dbo.Mesa AS m ON m.idMesa = v.idMesa 
                      INNER JOIN dbo.Eleccion AS elec ON elec.idEleccion = m.idEleccion
                      INNER JOIN dbo.Postulaciones post ON (post.idEleccion = elec.idEleccion AND post.DNI = ciu.DNI)
                      INNER JOIN dbo.Partido_Politico partPolitico ON post.idPartido = partPolitico.idPartido
WHERE     (v.idMesa IN
                          (SELECT     idMesa
                            FROM          dbo.Mesa AS m
                            WHERE      (idEleccion IN
                                                       (SELECT     TOP (5) e.idEleccion
                                                         FROM          dbo.Eleccion AS e INNER JOIN
                                                                                dbo.Eleccion_Cargo_Provincial AS ecp ON ecp.idEleccion = e.idEleccion
                                                         WHERE      (e.tipo = 'Cargo Provincial') AND (YEAR(e.Fecha) = YEAR(GETDATE()))
                                                         ORDER BY e.Fecha DESC))))
GROUP BY vc.DNI, partPolitico.Nombre, ciu.Nombre, ciu.Apellido, elec.Fecha
\end{lstlisting}

Creamos otra vista que sobre la vista anterior, calcula el porcentaje de votos de cada partido y lo filtra por quienes tienen mas de 20\%.

\begin{lstlisting}
SELECT     TOP (100) PERCENT FechaEleccion, PartidoPolitico, CantVotos * 100 / Cant_Total_Votos_Por_Eleccion AS Porcentaje
FROM         dbo.Ranking_Candidatos_Ultimas_5_Elecciones_A_Gobernador
WHERE     (CantVotos * 100 / Cant_Total_Votos_Por_Eleccion >= 20)
ORDER BY FechaEleccion, CantVotos DESC
\end{lstlisting}

Finalmente, consultamos la vista.

\begin{lstlisting}
	SELECT * FROM Partidos_Con_Mas_Del_20_Porciento_Ultimas_5_Gobernador
\end{lstlisting}

\end{enumerate}