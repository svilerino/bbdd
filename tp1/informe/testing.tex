\subsection{Codigo de testing de la solucion provista}
\fk{Testear triggers para validaciones}
--Utiles para testear triggers rapido
--ALTER TABLE Voto NOCHECK CONSTRAINT ALL
--ALTER TABLE Voto CHECK CONSTRAINT ALL

A continuacion se realizaran pruebas de solidez y correctitud de las restricciones implementadas:

\subsubsection{Jerarquia Disjuta NO SE}

Se realizan tests para verificar la implementacion de la restriccion: "Los campos tipo de las tablas con jerarquia disjunta deben contener los valores correctos".

\begin{enumerate}

\item \textbf{Eleccion:} La entidad eleccion posee un campo tipo, que particiona todas las entidades que heredan
de ella. Haremos un test para cada una.

\begin{enumerate}

\item Consulta Popular: Creamos una instancia de eleccion de tipo: "Consulta Popular" y verificamos el
resultado:

\end{enumerate}

\end{enumerate}

\subsubsection{Votos de una eleccion}

Verificaremos la conjuncion de las restricciones "Todos los votos para una eleccion son: o bien consulta popular o bien de tipo candidato según corresponda el tipo de eleccion",  "Si la eleccion es una consulta popular, los votos deben ser si/no. sino, deben ser candidatos'' y por ultimo, ''Un voto a candidato, debe referenciar a un candidato postulado para dicha eleccion''

\begin{enumerate}

\item \textbf{Consulta Popular:} Si la eleccion es una consulta popular, los votos cuya clave foranea
referencia esa eleccion deben ser de tipo ' A Plesbiscito Si ' o ' A Plesbiscito No' .

	\begin{itemize}
	\item 'A Plesbiscito Si': Creamos una elección de tipo 'Consulta Popular', e insertamos un voto
	de tipo 'A Plesbiscito Si'.
	\begin{lstlisting}
	SELECT SAPE
	\end{lstlisting}
	
	\item 'A Plesbiscito No': Creamos una elección de tipo 'Consulta Popular', e insertamos un voto
	de tipo 'A Plesbiscito No'.
	\begin{lstlisting}
	SELECT NO SAPE
	\end{lstlisting}
	\item 'Voto de tipo invalido': Creamos una 'Consulta Popular', e insertamos un voto
	de tipo 'A Plesbiscito Ni'.
	\begin{lstlisting}
	SELECT CAPAZ SAPE
	\end{lstlisting}
	
	\item 'Voto a un candidato': Creamos una 'Consulta Popular', e insertamos un voto
	de tipo 'A Candidato'.
	\begin{lstlisting}
	SELECT NI AHI SAPE
	\end{lstlisting}	
	
	\end{itemize}

\item \textbf{Eleccion a candidato:}  Si la eleccion es de tipo 'Cargo Federal', 'Cargo Municipal', Cargo Provincial' o tipo 'Cargo Legislativo', los votos cuya clave foranea referencia esa eleccion deben ser de tipo 'A Candidato', y deben referenciar a un candidato que este postulado a dicha eleccion

	\begin{itemize}
	\item 'Candidato Valido': Creamos una elección de tipo 'Cargo Federal', e insertamos un voto
	de tipo 'A candidato', cuya clave foranea a candidato referencia a un candidato cuyo DNI se encuentra en la tabla postulaciones para esa eleccion.
	\begin{lstlisting}
	SELECT ESTA BIEN
	\end{lstlisting}
	
	\item 'Candidato No Valido': Creamos una elección de tipo 'Cargo Provincial', e insertamos un voto
	de tipo 'A candidato', cuya clave foranea a candidato NO referencia a un candidato cuyo DNI se encuentra en la tabla postulaciones para esa eleccion.
	\begin{lstlisting}
	SELECT WILLIE
	\end{lstlisting}
		
	\item 'Voto a plesbiscito': Creamos una elección de tipo 'Cargo Provincial', e insertamos un voto
	de tipo 'A plesbiscito Si'.
	\begin{lstlisting}
	SELECT SAPE
	\end{lstlisting}
			
	\end{itemize}

\subsubsection{Multiples Candidatos:} Un partido politico puede presentar a un solo candidato por eleccion. Para chequear esto, vamos a crear una eleccion de tipo 'Cargo Federal', un partido politico
'Frente por la Derrota', y vamos a crear dos candidatos distintos y postularlos para la misma eleccion.
	\begin{lstlisting}
	SELECT CAMPORA
	\end{lstlisting}
			

\subsubsection{Multiples Mesas para un ciudadano:} Un partido politico puede presentar a un solo candidato por eleccion. Para chequear esto, vamos a crear una eleccion de tipo 'Cargo Federal', un partido politico
'Frente por la Derrota', y vamos a crear dos candidatos distintos y postularlos para la misma eleccion.
	\begin{lstlisting}
	SELECT UCR
	\end{lstlisting}
			

\subsubsection{Multiples Mesas para una  maquina:} Un ciudadano vota en una sola mesa por eleccion. Para chequear esto creamos una eleccion de tipo 'Consulta Popular', un ciudadano, dos mesas para esa eleccion, y asignamos al ciudadano en ambas mesas.
	\begin{lstlisting}
	SELECT PRO
	\end{lstlisting}
								
\subsubsection{Multiples Mesas para una maquina}:Una maquina funciona en una sola mesa por eleccion. Para chequear esto creamos una eleccion de tipo 'Consulta Popular', una maquina, dos mesas para esa eleccion, y asignamos a la maquina en ambas mesas.
	\begin{lstlisting}
	SELECT EL NENE
	\end{lstlisting}
	
\subsubsection{Validaciones de los votos}: Validaremos las restricciones: ''No permitir que la gente vote mas de una vez por eleccion'', ''El voto puede hacerse en la fecha de la eleccion de 8am a 6pm''. 
	\begin{itemize}
	\item 'Unico voto': Creamos una elección de tipo 'Consulta Popular', y hacemos que un ciudadano intente votar por segunda vez.
	\begin{lstlisting}
	SELECT IGNACIO UNBEKANT
	\end{lstlisting}
	
	\item 'Hora valida': Creamos una elección de tipo 'Consulta Popular', y hacemos que un ciudadano 		intente votar en un horario invalido.
	\begin{lstlisting}
	SELECT SAPE
	\end{lstlisting}	
	
	\end{itemize}					

\end{enumerate}
