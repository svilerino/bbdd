\subsection{Codigo de testing de la solucion provista}

En primera instancia, se realizaran pruebas de solidez y correctitud de las restricciones implementadas. Por último, tests de correctitud de las consultas requeridas por el enunciado.

\subsubsection{Votos de una eleccion}

Verificaremos la conjuncion de las restricciones "Todos los votos para una eleccion son: o bien consulta popular o bien de tipo candidato según corresponda el tipo de eleccion",  "Si la eleccion es una consulta popular, los votos deben ser si/no. sino, deben ser candidatos'' y por ultimo, ''Un voto a candidato, debe referenciar a un candidato postulado para dicha eleccion''

\begin{enumerate}

\item \textbf{Consulta Popular:} Si la eleccion es una consulta popular, los votos cuya clave foranea
referencia esa eleccion deben ser de tipo ' A Plesbiscito Si ' o ' A Plesbiscito No'. Creamos una elección de tipo 'Consulta Popular' utilizando la Stored Procedure 'Crear\_ Consulta\_ Popular' con los valores Descripcion: TEST y Fecha: '2015-06-09'. 

A continuacion, debemos crear una mesa. Para eso, debemos tener un presidente, un vicepresidente,
un centro y una maquina(y para ella, un tecnico). Las creamos ejecutando:

\begin{lstlisting}

INSERT INTO Ciudadano VALUES
 (1,'JUAN','VOTANTE'),
 (2,'PEPE','PRESIDENTE'),
 (3,'VICTOR','VICEPRESIDENTE'),
 (4,'TOMAS','TECNICO'),
 (5,'CARLOS', 'CANDIDATO');

 INSERT INTO Centro VALUES
 (1,'Escuela', '9 de julio 1');

 INSERT INTO Maquina VALUES
 (1,4)

 INSERT INTO Mesa VALUES
 (1,1,2,3,1,1,1)
\end{lstlisting}

	\begin{itemize}
	\item 'A Plesbiscito Si': Insertar un voto 'A Plesbiscito Si' deberia realizarse correctamente.
	Ejecutamos la Stored Procedure 'Votar\_ Consulta\_ Popular' con los parametros DNIdelVotante=1, idMesaDelVoto=1 y OpcionVoto=1. Acto seguido, ejecutamos:
	\begin{lstlisting}
	SELECT * FROM Voto
	\end{lstlisting}
	Y efectivamente esto nos da como resultado:	
	\vspace{2mm}
	
	\begin{tabular}{| l| l| l| l| }
	\hline 
	   & idVoto & idMesa & Tipo \\
	   \hline
	 1 & 1 & 1 & A Plesbiscito Si \\
 	 \hline 
	\end{tabular}
	
	\vspace{2mm}
	\item 'A Plesbiscito No': Insertamos un voto de tipo 'A Plesbiscito No', lo cual deberia ejecutarse correctamente, usando la Stored Procedure 'Votar\_ Consulta\_ Popular' con los parametros DNIdelVotante=1, idMesaDelVoto=1 y OpcionVoto=0. Resultado:
	
	\begin{tabular}{| l| l| l| l| }
	\hline 
	   & idVoto & idMesa & Tipo \\
	   \hline
	 1 & 1 & 1 & A Plesbiscito No \\
 	 \hline 
	\end{tabular}
	
	\vspace{2mm}	
	
	\item 'Voto de tipo invalido': Insertamos un voto de tipo 'A Plesbiscito No', lo cual NO deberia ejecutarse correctamente, usando la Stored Procedure 'Votar\_ Consulta\_ Popular' con los parametros DNIdelVotante=1, idMesaDelVoto=1 y OpcionVoto=3 (Esto representa un tipo de voto invalido). Resultado:
	
\begin{tabular}{| l| l| l| l| }
	\hline 
	   & idVoto & idMesa & Tipo \\
	   \hline
	 &  &  &  \\
 	 \hline 
	\end{tabular}
	
	\vspace{2mm}	
		
	
	\item 'Voto a un candidato': Insertamos un voto 'A Candidato' lo cual NO deberia ejecutarse correctamente, usando la Stored Procedure 'Votar\_ Candidato' sobre la Consulta Popular, con los parametros DNIdelVotante=1, idMesaDelVoto=1 y DniCandidato=5. Inmediatamente recibimos el aviso:
	''La eleccion es para Consulta Popular pero el voto no es a plesbicito''. y como resultado:
		
\begin{tabular}{| l| l| l| l| }
	\hline 
	   & idVoto & idMesa & Tipo \\
	   \hline
	 &  &  &  \\
 	 \hline 
	\end{tabular}
	
	\vspace{2mm}	
	\end{itemize}
\item \textbf{Eleccion a candidato:}  Si la eleccion es de tipo 'Cargo Federal', 'Cargo Municipal', Cargo Provincial' o tipo 'Cargo Legislativo', los votos cuya clave foranea referencia esa eleccion deben ser de tipo 'A Candidato', y deben referenciar a un candidato que este postulado a dicha eleccion

	\begin{itemize}
	\item 'Candidato Valido': Creamos una elección de tipo 'Cargo Federal', e insertamos un voto
	de tipo 'A candidato', cuya clave foranea a candidato referencia a un candidato cuyo DNI se encuentra en la tabla postulaciones para esa eleccion.
	\begin{lstlisting}
	SELECT ESTA BIEN
	\end{lstlisting}
	
	\item 'Candidato No Valido': Creamos una elección de tipo 'Cargo Provincial', e insertamos un voto
	de tipo 'A candidato', cuya clave foranea a candidato NO referencia a un candidato cuyo DNI se encuentra en la tabla postulaciones para esa eleccion.
	\begin{lstlisting}
	SELECT WILLIE
	\end{lstlisting}
		
	\item 'Voto a plesbiscito': Creamos una elección de tipo 'Cargo Provincial', e insertamos un voto
	de tipo 'A plesbiscito Si'.
	\begin{lstlisting}
	SELECT SAPE
	\end{lstlisting}
			
	\end{itemize}
\end{enumerate}
\subsubsection{Multiples Candidatos:} Un partido politico puede presentar a un solo candidato por eleccion. Para chequear esto, vamos a crear una eleccion de tipo 'Cargo Federal', un partido politico
'Frente por la Derrota', y vamos a crear dos candidatos distintos y postularlos para la misma eleccion.
	\begin{lstlisting}
	SELECT CAMPORA
	\end{lstlisting}
			

\subsubsection{Multiples Mesas para un ciudadano:} Un partido politico puede presentar a un solo candidato por eleccion. Para chequear esto, vamos a crear una eleccion de tipo 'Cargo Federal', un partido politico
'Frente por la Derrota', y vamos a crear dos candidatos distintos y postularlos para la misma eleccion.
	\begin{lstlisting}
	SELECT UCR
	\end{lstlisting}
			

\subsubsection{Multiples Mesas para una  maquina:} Un ciudadano vota en una sola mesa por eleccion. Para chequear esto creamos una eleccion de tipo 'Consulta Popular', un ciudadano, dos mesas para esa eleccion, y asignamos al ciudadano en ambas mesas.
	\begin{lstlisting}
	SELECT PRO
	\end{lstlisting}
								
\subsubsection{Multiples Mesas para una maquina}:Una maquina funciona en una sola mesa por eleccion. Para chequear esto creamos una eleccion de tipo 'Consulta Popular', una maquina, dos mesas para esa eleccion, y asignamos a la maquina en ambas mesas.
	\begin{lstlisting}
	SELECT EL NENE
	\end{lstlisting}
	
\subsubsection{Validaciones de los votos}: Validaremos las restricciones: ''No permitir que la gente vote mas de una vez por eleccion'', ''El voto puede hacerse en la fecha de la eleccion de 8am a 6pm''. 
	\begin{itemize}
	\item 'Unico voto': Creamos una elección de tipo 'Consulta Popular', y hacemos que un ciudadano intente votar por segunda vez.
	\begin{lstlisting}s
	SELECT IGNACIO UNBEKANT
	\end{lstlisting}
	
	\item 'Hora valida': Creamos una elección de tipo 'Consulta Popular', y hacemos que un ciudadano 		intente votar en un horario invalido.
	\begin{lstlisting}
	SELECT SAPE
	\end{lstlisting}	
	
	\end{itemize}					

\subsubsection{Consultas del enunciado}:



